%%%%%%%%%%%%%%%%%%%%%%%%%%%%%%%%%%%%%%%%%%%%%%%%%%%%%%%%%%%%%%%%
%                                                              %
% Jayson Haddon                                                %
% ECE 351 Section 51                                           %
% Lab 7                                                        %
% 03/08/2022    
%Github https://github.com/haddjays?tab=repositories
% Necessary Details:                            %
%                                                              %
%%%%%%%%%%%%%%%%%%%%%%%%%%%%%%%%%%%%%%%%%%%%%%%%%%%%%%%%%%%%%%%%
\documentclass[11pt,a4]{report}
\usepackage[english]{babel}
%\usepackage{natbib}
\usepackage{url}
\usepackage[utf8x]{inputenc}
\usepackage{amsmath}
\usepackage{graphicx}
\graphicspath{{images/}}
\usepackage{parskip}
\usepackage{fancyhdr}
\usepackage{vmargin}
\usepackage{listings}
\usepackage{hyperref}
\usepackage{xcolor}
\definecolor{codegreen}{rgb}{0,0.6,0}
\definecolor{codegray}{rgb}{0.5,0.5,0.5}
\definecolor{codeblue}{rgb}{0,0,0.95}
\definecolor{backcolour}{rgb}{0.95,0.95,0.92}

\lstdefinestyle{mystyle}{
    backgroundcolor=\color{backcolour},   
    commentstyle=\color{codegreen},
    keywordstyle=\color{codeblue},
    numberstyle=\tiny\color{codegray},
    stringstyle=\color{codegreen},
    basicstyle=\ttfamily\footnotesize,
    breakatwhitespace=false,         
    breaklines=true,                 
    captionpos=b,                    
    keepspaces=true,                 
    numbers=left,                    
    numbersep=5pt,                  
    showspaces=false,                
    showstringspaces=false,
    showtabs=false,                  
    tabsize=2
}
 
\lstset{style=mystyle}

\setmarginsrb{3 cm}{2.5 cm}{3 cm}{2.5 cm}{1 cm}{1.5 cm}{1 cm}{1.5 cm}

\title{Lab 7}								
% Title
\author{Jayson Haddon}						
% Author
\date{08/27/2020}
% Date

\makeatletter
\let\thetitle\@title
\let\theauthor\@author
\let\thedate\@date
\makeatother

\pagestyle{fancy}
\fancyhf{}
\rhead{\theauthor}
\lhead{\thetitle}
\cfoot{\thepage}
%%%%%%%%%%%%%%%%%%%%%%%%%%%%%%%%%%%%%%%%%%%%
\begin{document}

%%%%%%%%%%%%%%%%%%%%%%%%%%%%%%%%%%%%%%%%%%%%%%%%%%%%%%%%%%%%%%%%%%%%%%%%%%%%%%%%%%%%%%%%%

\begin{titlepage}
	\centering
    \vspace*{0.5 cm}
\begin{center}    \textsc{\Large   ECE 351 }\\[2.0 cm]	\end{center}% University Name
	\textsc{\Large Block Diagrams and System Stability  }\\[0.5 cm]				% Course Code
	\rule{\linewidth}{0.2 mm} \\[0.4 cm]
	{ \huge \bfseries \thetitle}\\
	\rule{\linewidth}{0.2 mm} \\[1.5 cm]
	
	\begin{minipage}{0.4\textwidth}
		\begin{flushleft} \large
		%	\emph{Submitted To:}\\
		%	Name\\
          % Affiliation\\
           %contact info\\
			\end{flushleft}
			\end{minipage}~
			\begin{minipage}{0.4\textwidth}
            
			\begin{flushright} \large
			\emph{Submitted By :} \\
			Jayson Haddon  
		\end{flushright}
           
	\end{minipage}\\[2 cm]
	
%	\includegraphics[scale = 0.5]{PICMathLogo.png}
    
    
    
    
	
\end{titlepage}

%%%%%%%%%%%%%%%%%%%%%%%%%%%%%%%%%%%%%%%%%%%%%%%%%%%%%%%%%%%%%%%%%%%%%%%%%%%%%%%%%%%%%%%%%
\tableofcontents
\pagebreak

\renewcommand{\thesection}{\arabic{section}}
\section{Introduction}
The purpose of this lab was to find the poles and zeros of a Laplace domain block diagram and use the transfer function to judge system stability.
\section{Equations}
The two differential equations that will be used for this lab are shown in equation 1 and 2 shown below. 
\begin{equation}
    G(s) = \frac{s+9}{(s^2-6s-16)(s+4)}
\end{equation}
\begin{equation}
    A(s) = \frac{s+4}{s^2+4s+3}
\end{equation}
\begin{equation}
  B(s) = s^2+26s+168
\end{equation}


\section{Methodology}
The first task in part one was to type G(S), A(s), and B(s) in factored form and defining what the poles and zeros are. To do this I used the quadratic formula to factor the numerator and denominator of each function. I then check my answer using the signal.tf2zpk function. My code, the output of my code and the expressions can be seen in results section 1. 

The third task was to type the open loop transfer function. To complete this I choose the fastest loop through the block diagram. This ended up to be A(s) * G(s). The proper equation can be seen below in the results task 3 and the answer to task 4 can be seen in the questions section. 

The firth task was to plot the step response of the open loop transfer function. To complete this I used the convolve function to convolve the numerator of a and the numerator of g and then convolve the denominator of a with the denominator of g. I then used the signal.step function to graph the two convolutions.This plot can be seen below in the results. 

The sixth task of part 1 can be seen below in the question section.

The first task in part two was to type the closed loop transfer function symbolically. To accomplish this I used the feed back theorem for block diagrams to find the transfer function. The transfer function can be seen below in the results part 2. 

The second task of part two was to find the numerical value for the numerator and denominator of the transfer function. 

The third task of part two can be seen below in the question section.

The fourth task of part two was to plot the step response of the closed loop transfer function. To do this I used the numerator and denominator expression I created in task one part two in the scipy.signal.step function. The code and plot can be seen in the results below. 

The fifth task of part two can be seen below in the question section.

\section{Results}


\subsection{Part 1 Task 1}

\begin{lstlisting}[language=Python]
#....................Defined a(s)...........#
numa = [1,4]
print('A Num =', numa)

dena = sig.convolve([1,1],[1,3])
print('A Den =', dena)

print('Zeros of A = s+4' )
print('Poles of A = (s+1)(s+3)' )

ZA, PA, KA = sig.tf2zpk(numa, dena)

print('sig.tf2zpk method Zeros of A=', ZA)
print('sig.tf2zpk method Pole of A=', PA)



#...................Define g(s)...................#

numg = [1,9]
print('g Num =', numg)

deng = sig.convolve([1,-6,-16],[1,4])
print('g Den =', deng)

print('Zeros of G = s+9' )
print('Poles of G = (s-8)(s+2)(s+4)' )

ZG, PG, KG = sig.tf2zpk(numg, deng)

print('sig.tf2zpk method Zeros of G=', ZG)
print('sig.tf2zpk method Pole of G=', PG)

#................Define B(s).................#

numb = [1,26,168]
print('B Num =', numb)

print('Zeros of B = (s+12)(s+14)' )

ZB = np.roots(numb)

print(' Roots Method Zeros of  B=', ZB)
\end{lstlisting}

\begin{lstlisting}
A Num = [1, 4]
A Den = [1 4 3]
Zeros of A = s+4
Poles of A = (s+1)(s+3)
sig.tf2zpk method Zeros of A= [-4.]
sig.tf2zpk method Pole of A= [-3. -1.]
g Num = [1, 9]
g Den = [  1  -2 -40 -64]
Zeros of G = s+9
Poles of G = (s-8)(s+2)(s+4)
sig.tf2zpk method Zeros of G= [-9.]
sig.tf2zpk method Pole of G= [ 8. -4. -2.]
B Num = [1, 26, 168]
Zeros of B = (s+12)(s+14)
 Roots Method Zeros of  B= [-14. -12.]
\end{lstlisting}
From these results the following equations can be made
\begin{equation}
    A(s) = \frac{s+4}{(s+1)(s+3}
\end{equation}
\begin{equation}
    G(s) = \frac{s+9}{(s-8)(s+2)(s+4)}
\end{equation}
\begin{equation}
    B(s) = (s+12)(s+14)
\end{equation}


\subsection{Part 1 Task 3}

\begin{equation}
    H(s) = \frac{Y(s)}{X(s)} = \frac{s+9}{(s+1)(s+3)(s-8)(s+2)}
\end{equation}

\subsection{Part 1 Task 5}
\begin{lstlisting}
#.............Define Open Loop Y(s)/X(s)...................#

OLnum = sig.convolve(numa,numg)
OLden = sig.convolve(dena,deng)

tout, yout = sig.step((OLnum,OLden),T=t)

plt.figure(figsize = (10, 12))
plt.subplot(3, 1, 1)
plt.plot(tout,yout)
plt.grid()
plt.ylabel('Amplitude ')
plt.xlabel( 't')
plt.title('Scipy signal Open Loop')
plt.show

\end{lstlisting}

\begin{figure}[h!]
    \begin{center}
  \caption{Graph of Open Loop Function}
  \includegraphics[scale=0.7]{Part 1 Task 5.png}
\end{center}
\end{figure}
\newpage
\subsection{Part 2 Task 1}

\begin{equation}
    H(s) = Y(s)/X(s) = \frac{numa*numg}{dena*numb*numg+deng*dena}
\end{equation}

\subsection{Part 2 Task 2}
\begin{lstlisting}[language=Python]
num1 = sig.convolve(numa, numg)
den1 = sig.convolve(deng, dena) 
den2a = sig.convolve(dena, numb)
den2b = sig.convolve(den2a, numg)
denF = den1 + den2b
print(num1)
print(den1)
print(denF)
\end{lstlisting}

\begin{lstlisting}
[ 1 13 36]
[   1    2  -45 -230 -376 -192]
[   2   41  500 2995 6878 4344]
\end{lstlisting}
From this the transfer function can be seen to be
\begin{equation}
    H(s) = Y(s)/X(s) = \frac{s^2+13s+36}{2s^5+41s^4+500s^3+995s^2+6878s+ 4344}
\end{equation}


\subsection{Part 2 Task 4}
\begin{lstlisting}[language=Python]
tout1, yout1 = sig.step((num1, denF), T=t)

plt.figure(figsize = (10, 12))
plt.subplot(3, 1, 2)
plt.plot(tout1,yout1)
plt.grid()
plt.ylabel('Amplitude ')
plt.xlabel( 't')
plt.title('Scipy signal impulse')
plt.show
\end{lstlisting}

\begin{figure}[h!]
    \begin{center}
  \caption{Step Impulse Closed Loop}
  \includegraphics[scale=0.6]{Part 2 Task 4.png}
\end{center}
\end{figure}





\section{Error Analysis}
This lab took me a little while to figure out how useful the convolve function could be and how to utilize it. Once I figured out how to utilize the convolve function, it become very easy with little error to symbolically solve all the transfer functions. 
\section{Questions}
\subsection{Part 1 Task 4}
No, the open loop is not stable. The open loop is not stable because of the s-8 in the denominator. This will cause a positive exponential function which will make the system unstable. 
\subsection{Part 1 Task 6}
My results from Task 5 support my answer from task 4. This is because I said that the system was unstable because of the s-8 which would cause a positive exponential. This can be seen in the plot from task 5 because the system does not become stable and keeps exponentially increasing thus showing that the system is unstable.. 
\subsection{Part 2 Task 3}
The close loop function will be stable. This is because all the poles in the are positive which will lead to negative exponential functions. This will make the system stable by not going off into infinity. 
\subsection{Part 2 Task 5}
Yes, my results from task 4 support my answer from task 3. I task 3 I said that the system was stable because all the poles were positive. In the figure from task 4 it can be seen that the graph levels out at around 0.008 and continues to approach that number as time goes on. This shows that the system is stable. 

\subsection{End of Report Questions}
1. In Part 1 Task 5, why does convolving the factored terms using scipy.signal.convolve()
result in the expanded form of the numerator and denominator? Would this work with your
user-defined convolution function from Lab 3? Why or why not?

No, I don't think this would work because the user defined functions in lab 3 were in the time domain while the functions that we are doing the operations in for this lab are in the s domain. 

2. Discuss the difference between the open- and closed-loop systems from Part 1 and Part 2.
How does stability differ for each case, and why?

The open loop system for part 1 was the quickest method through the function. This will result in some amount of error because you are only looking at the quickest path. Since you are only looking at one path through the function, the stability for the one path that you looked at could be off. For the closed loop system you are looking at the entire system. This will result in less error from the open loop because you are looking at the whole system and could result in a different stability. The stability for the open loop was unstable while for the closed loop is was stable. 

3. What is the difference between scipy.signal.residue() used in Lab 6 and
scipy.signal.tf2zpk() used in this lab?

The scipy.signal.tf2zpk takes a function in the s domain and factors the numerator and denominator. This gives you the roots and poles of the s domain function. The scipy.signal.residue gives you the coefficients and order of the system in the time domain from a function in the s domain. The tf2zpk factors a fraction while the residue takes the inverse Laplace transfer of a fraction.

4. Is it possible for an open-loop system to be stable? What about for a closed-loop system to
be unstable? Explain how or how not for each.

Yes, it would be possible for an open loop system to be stable and for a closed loop system to be unstable. It all depends on the design of system. If the open loop for a system was designed right it could be a stable system and if the closed loop of a system was designed poorly or it was wanted to be unstable it could be. 

5. Leave any feedback on the clarity/usefulness of the purpose, deliverables, and expectations
for this lab.

I have no feedback for this lab. 

\section{Conclusion}
In conclusion, from this lab I was able to see how the stability of an open and closed loop system by finding the transfer function and graphing it. I was also able to use tools to factor an expression to find the poles and zeros and then use those poles and zeros to see if the system would be stable or not. I was then able to use the factor expression with the convolve function to graph the step response of the system over time. 


\section{Github}
https://github.com/haddjays?tab=repositories
\end{document}

\section{Conclusion}

%This template was created by Roza Aceska.
