%%%%%%%%%%%%%%%%%%%%%%%%%%%%%%%%%%%%%%%%%%%%%%%%%%%%%%%%%%%%%%%%
%                                                              %
% Jayson Haddon                                                %
% ECE 351 Section 51                                           %
% Lab 2                                                        %
% 02/01/2022    
%Github https://github.com/haddjays?tab=repositories
% Necessary Details:                            %
%                                                              %
%%%%%%%%%%%%%%%%%%%%%%%%%%%%%%%%%%%%%%%%%%%%%%%%%%%%%%%%%%%%%%%%
\documentclass[11pt,a4]{report}
\usepackage[english]{babel}
%\usepackage{natbib}
\usepackage{url}
\usepackage[utf8x]{inputenc}
\usepackage{amsmath}
\usepackage{graphicx}
\graphicspath{{images/}}
\usepackage{parskip}
\usepackage{fancyhdr}
\usepackage{vmargin}
\usepackage{listings}
\usepackage{hyperref}
\usepackage{xcolor}
\definecolor{codegreen}{rgb}{0,0.6,0}
\definecolor{codegray}{rgb}{0.5,0.5,0.5}
\definecolor{codeblue}{rgb}{0,0,0.95}
\definecolor{backcolour}{rgb}{0.95,0.95,0.92}

\lstdefinestyle{mystyle}{
    backgroundcolor=\color{backcolour},   
    commentstyle=\color{codegreen},
    keywordstyle=\color{codeblue},
    numberstyle=\tiny\color{codegray},
    stringstyle=\color{codegreen},
    basicstyle=\ttfamily\footnotesize,
    breakatwhitespace=false,         
    breaklines=true,                 
    captionpos=b,                    
    keepspaces=true,                 
    numbers=left,                    
    numbersep=5pt,                  
    showspaces=false,                
    showstringspaces=false,
    showtabs=false,                  
    tabsize=2
}
 
\lstset{style=mystyle}

\setmarginsrb{3 cm}{2.5 cm}{3 cm}{2.5 cm}{1 cm}{1.5 cm}{1 cm}{1.5 cm}

\title{Lab 2}								
% Title
\author{Jayson Haddon}						
% Author
\date{08/27/2020}
% Date

\makeatletter
\let\thetitle\@title
\let\theauthor\@author
\let\thedate\@date
\makeatother

\pagestyle{fancy}
\fancyhf{}
\rhead{\theauthor}
\lhead{\thetitle}
\cfoot{\thepage}
%%%%%%%%%%%%%%%%%%%%%%%%%%%%%%%%%%%%%%%%%%%%
\begin{document}

%%%%%%%%%%%%%%%%%%%%%%%%%%%%%%%%%%%%%%%%%%%%%%%%%%%%%%%%%%%%%%%%%%%%%%%%%%%%%%%%%%%%%%%%%

\begin{titlepage}
	\centering
    \vspace*{0.5 cm}
\begin{center}    \textsc{\Large   ECE 351 }\\[2.0 cm]	\end{center}% University Name
	\textsc{\Large User-Defined Functions  }\\[0.5 cm]				% Course Code
	\rule{\linewidth}{0.2 mm} \\[0.4 cm]
	{ \huge \bfseries \thetitle}\\
	\rule{\linewidth}{0.2 mm} \\[1.5 cm]
	
	\begin{minipage}{0.4\textwidth}
		\begin{flushleft} \large
		%	\emph{Submitted To:}\\
		%	Name\\
          % Affiliation\\
           %contact info\\
			\end{flushleft}
			\end{minipage}~
			\begin{minipage}{0.4\textwidth}
            
			\begin{flushright} \large
			\emph{Submitted By :} \\
			Jayson Haddon  
		\end{flushright}
           
	\end{minipage}\\[2 cm]
	
%	\includegraphics[scale = 0.5]{PICMathLogo.png}
    
    
    
    
	
\end{titlepage}

%%%%%%%%%%%%%%%%%%%%%%%%%%%%%%%%%%%%%%%%%%%%%%%%%%%%%%%%%%%%%%%%%%%%%%%%%%%%%%%%%%%%%%%%%
\tableofcontents
\pagebreak

\renewcommand{\thesection}{\arabic{section}}
\section{Introduction}
The objective of this lab was to create user-defined functions in python and use them to demonstrate various signal operations. These signal operations are time shifting, time scaling, signal addition, and discrete differentiation. Some useful background information is that python relies heavily on python lists and numpy arrays. You can use for and if/else loops. The len() function is often used in for loops to return the number of elements. All plots need to have x and y axis labeled. 
\section{Equations}
The two important equations for this lab is the ramp and step function.
\newline
\newline
 r(t) =
\[ \begin{cases} 
      x & x\geq 0 \\
      0 & < 0 
   \end{cases}
\]
\newline
u(t)=
\[ \begin{cases} 
      1 & t\geq 0 \\
      0 & t <  0 
   \end{cases}
\]
\section{Methodology}
The first task in part 4 asks to create a function func1 that implements the function y = cos(t). To complete this I modified the example code by removing the if else loop and then by just setting y(t) = np.cos(t[i]). The code for this task can be seen in the result section under part 4. The graph of the cos function can be seen below in figure 2. The next part of the lab asks to plot the graph shown below in figure 1.

[
\begin{figure}[h!]
    \begin{center}
  \caption{Task for Part 5}
  \includegraphics[scale=0.5]{Lab 2 sample graph.PNG}
\end{center}
\end{figure} ]
\newpage

The first part of part 5 ask to derive an equation for the graph shown in figure 1. To make the equation I used the ramp and step function that was taught in class to create the shape shown above. The equation can be seen in equation 1 in the result section. The next part of task 5 ask to create user function to plot the sample graph in python. To do this I created two different functions. One was the ramp function and the other was the step function. In the ramp function I used the definition of the ramp function to define that when t > 0 y = t and when t < 0 y = 0. This would allow for time shifting and create the slope of one that is the ramp function. For the step function I also used the definition of the step function to define that when t > 0 y = 1 and when t < 0 y = 0. After creating the step and ramp function, I used the equation I derived for the plot and graphed the plot using y given functions. The code and plot for this can be seen in part 5 of the result section. 

The last part of the lab was time shifting. The first task for part 6 was to apply a time reversal and plot the result. To do this I inverted the t in all of my functions. This equation and plot can be seen below in the result section under part 6 task 1. The next task was to apply a time shift of f(t-4) and f(-t-4). To do this I applied the time shift inside of my call to the functions. The equations and plots can be seen under results section part 6 task 2. The third task was to time shift the functions by f(t/2) and f(t*2). To do this I applied the time shift inside of my call to the functions. The equations and plots can be seen below in results under part 6 task 3. The next task was to graph the derivative of the function is part 4 task 2 by hand. This can be seen in results under part 6 task 4. The last task was to take the derivative of the function. To do this the np.diff function was used. To get the arrayies to be the same size and to take the derivative I first took the derivative of time and then took the derivative of my function over the derivative of time. I then plotted the function. The code and plot can be seen below in the results under Part 6 task 5. 


\section{Results}
\subsection{Part 4}
\begin{lstlisting}[language=Python]
import numpy as np
import matplotlib.pyplot as plt

#plt.rcParams.update({'fontsize': 14}) # Set font size in plots

steps = 1e-2 # Define step size
t = np.arange(0, 10 + steps , steps) # Add a step size to make sure the
# plot includes 5.0. Since np.arange () only
# goes up to , but doesnt include the
# value of the second argument
print('Number of elements: len(t) = ', len(t), '\nFirst Element: t[0] = ', t[0], 
      ' \nLast Element: t[len(t) - 1] = ', t[len(t) - 1])
# Notice the array might be a different size than expected since Python starts
 # at 0. Then we will use our knowledge of indexing to have Python print the
# first and last index of the array. Notice the array goes from 0 to len() - 1

# --- User - Defined Function ---
# Create output y(t) using a for loop and if/else statements
def func1(t): # The only variable sent to the function is t
    y = np.zeros(t.shape)
    
    for i in range(len(t)):
            y[i] = np.cos(t[i]) #create a cosine graph
    return y

y = func1(t) # call the function we just created

plt.figure(figsize = (10, 7))
plt.subplot(2, 1, 1)
plt.plot(t, y)
plt.grid()
plt.ylabel('Amplitude')
plt.xlabel( 't')
plt.title('Part 4 Cosine Graph ')



\end{lstlisting}

\begin{figure}[h!]
    \begin{center}
  \caption{Part 4 Cosine Graph}
  \includegraphics[scale=0.5]{cosine graph.png}
\end{center}
\end{figure}
\newpage
\subsection{Part 5}
\begin{equation}
    y(t) = 1r(t) -r(t-3) + 5u(t-3)-2u(t-6) - 2r(t-6)
\end{equation}

\begin{lstlisting}[language=Python]
def ramp(t):
    y = np.zeros(t.shape)
    
    for i in range(len(t)):
        if (t[i] > 0):
            y[i]=t[i]
        else: 
            y[i] = 0
    
    return y
        
def step(t):
    x = np.zeros(t.shape)
    
    for i in range(len(t)):
        if (t[i] > 0):
            x[i]= 1
        else:
            y[i]=0
    return x

y = ramp(t) # Ramp function
x = step(t) # Step function

z = ramp(t) - ramp(t-3) +5*step(t-3)-2*step(t-6)-2*ramp(t-6)

plt.figure(figsize = (10, 7))
plt.subplot(2, 1, 1)
plt.plot(t, z)
plt.grid()
plt.ylabel('Amplitude')
plt.xlabel( 't')
plt.title('Part 4 Cosine Graph ')
\end{lstlisting}

\begin{figure}[h!]
    \begin{center}
  \caption{Part 5}
  \includegraphics[scale=0.5]{Part 2.png}
\end{center}
\end{figure}
\newpage

\subsection{Part 6}

\subsubsection{Task 1}


\begin{equation}
    y(t) = 1r(-t) -r(-t-3) + 5u(-t-3)-2u(-t-6) - 2r(-t-6)
\end{equation}

\begin{figure}[h!]
    \begin{center}
  \caption{Part 6 Task 1}
  \includegraphics[scale=0.5]{Part 6 task 1.png}
\end{center}
\end{figure}

\subsubsection{Task 2}


\begin{equation}
    y(t) = 1r(t-4) -r(t-3-4) + 5u(t-3-4)-2u(t-6-4) - 2r(t-6-4)
\end{equation}

\begin{equation}
    y(t) = 1r(-t-4) -r(-t-3-4) + 5u(-t-3-4)-2u(-t-6-4) - 2r(-t-6-4)
\end{equation}

\begin{figure}[h!]
    \begin{center}
  \caption{Part 6 Task 2 f(t-4)}
  \includegraphics[scale=0.5]{Part 6 task 2 a.png}
\end{center}
\end{figure}
\newpage
\begin{figure}[h!]
    \begin{center}
  \caption{Part 6 Task 2 f(-t-4)}
  \includegraphics[scale=0.5]{Part 6 task 2 b.png}
\end{center}
\end{figure}

\subsubsection{Task 3}

\begin{equation}
    y(t) = 1r(t/2) -r((t-3)/2) + 5u((t-3)/2)-2u((t-6)/2) - 2r((t-6)/4)
\end{equation}

\begin{equation}
    y(t) = 1r(t*2) -r((t-3)*2) + 5u((t-3)/2)-2u((t-6)*2) - 2r((t-6)*4)
\end{equation}

\begin{figure}[h!]
    \begin{center}
  \caption{Part 6 Task 3 f(t/2)}
  \includegraphics[scale=0.5]{Part 6 task 3 a.png}
\end{center}
\end{figure}

\begin{figure}[h!]
    \begin{center}
  \caption{Part 6 Task 3 f(t*2)}
  \includegraphics[scale=0.5]{Part 6 task 3 b.png}
\end{center}
\end{figure}
\newpage
\subsubsection{Task 4}

\begin{figure}[h!]
    \begin{center}
  \caption{Part 6 Task 4 HandDrawn}
  \includegraphics[scale=0.5]{handdrawn.PNG}
\end{center}
\end{figure}

\subsubsection{Task 5}

\begin{lstlisting}[language=Python]

def ramp(t):
    y = np.zeros(t.shape)
    
    for i in range(len(t)):
        if (t[i] > 0):
            y[i]=t[i]
        else: 
            y[i] = 0
    return y
        
def step(t):
    x = np.zeros(t.shape)
    
    for i in range(len(t)):
        if (t[i] > 0):
            x[i]= 1
        else:
            x[i]=0
            
    return x

def deriv(t):
    u = np.zeros(t.shape)
    
    u = ramp(t) - ramp((t-3)) +5*step((t-3))-2*step((t-6))-2*ramp((t-6))
    return u
u = deriv(t)

y = ramp(t) # Ramp function
x = step(t) # Step function
z = ramp(t) - ramp((t-3)) +5*step((t-3))-2*step((t-6))-2*ramp((t-6))

dt = np.diff(t)
dy = np.diff((deriv(t)))/dt

plt.figure(figsize = (10, 7))
plt.subplot(2, 1, 1)
plt.ylim(-2,10)
plt.plot(t, u)
plt.plot(t[range(len(dy))],dy)
plt.grid()
plt.ylabel('Amplitude')
plt.xlabel( 't')
plt.title('Part 6 ')
plt.show


\end{lstlisting}

\begin{figure}[h!]
    \begin{center}
  \caption{Part 5}
  \includegraphics[scale=0.5]{derivative.png}
\end{center}
\end{figure}

\section{Error Analysis}

The difficulties I had during this experiment was first trying to understand how to graph the ramp and step function and then taking the derivative of the plot. Taking the derivative was especially difficult and I had to do quite a bit of searching before I got even get a start on taking the derivative. 

\section{Questions}
1.) Are the plots from Part 3 Task 4 and Part 3 Task 5 identical? Is it possible for them to
match? Explain why or why not. 

No the plots from Task 4 and Task 5 are not identical. The plots are not identical because when you take the derivative of something that has an infinite slope, you have a hole at that instances. The np.diff function still models that slope as going to infinity. In a sense the graphs are the same but they are not because the hand drawn has a hole, while the computer simulation goes to infinite. 

2.) How does the correlation between the two plots (from Part 3 Task 4 and Part 3 Task 5)
change if you were to change the step size within the time variable in Task 5? Explain why
this happens

If you were to change the steps size you might be able to get to the point where the simulation treads the infinite slopes as holes and then the two plots would be the same. This would be because as you took big time steps, you would be jumping over the infinite slope.

3.) Leave any feedback on the clarity of lab tasks, expectations, and deliverable.

My only questions is if I am doing the lab report to the correct standards. 


\section{Conclusion}
From this lab, I learned how to write functions in python. I also got a better understanding of what a step and ramp function is and how to use them. It was really useful to see how you actually plot the problems that we have been doing in class. It was interesting how the time shifts could be used and how to use the step and ramp functions to create certain shapes. I am still a little confused on what is happening for the derivative function. I feel the lab was successful for it widened my knowledge on what we are learning in class. I feel like I could have wrote my code in a better manner. 



\section{Github}
https://github.com/haddjays?tab=repositories
\end{document}

\section{Conclusion}

%This template was created by Roza Aceska.
