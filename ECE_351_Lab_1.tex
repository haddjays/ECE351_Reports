%%%%%%%%%%%%%%%%%%%%%%%%%%%%%%%%%%%%%%%%%%%%%%%%%%%%%%%%%%%%%%%%
%                                                              %
% Jayson Haddon                                                %
% ECE 351 Section 51                                           %
% Lab 1                                                        %
% 01/25/2022    
%Github https://github.com/haddjays?tab=repositories
% Necessary Details:  So much fun.                             %
%                                                              %
%%%%%%%%%%%%%%%%%%%%%%%%%%%%%%%%%%%%%%%%%%%%%%%%%%%%%%%%%%%%%%%%
\documentclass[11pt,a4]{report}
\usepackage[english]{babel}
%\usepackage{natbib}
\usepackage{url}
\usepackage[utf8x]{inputenc}
\usepackage{amsmath}
\usepackage{graphicx}
\graphicspath{{images/}}
\usepackage{parskip}
\usepackage{fancyhdr}
\usepackage{vmargin}
\usepackage{listings}
\usepackage{hyperref}
\usepackage{xcolor}
\definecolor{codegreen}{rgb}{0,0.6,0}
\definecolor{codegray}{rgb}{0.5,0.5,0.5}
\definecolor{codeblue}{rgb}{0,0,0.95}
\definecolor{backcolour}{rgb}{0.95,0.95,0.92}

\lstdefinestyle{mystyle}{
    backgroundcolor=\color{backcolour},   
    commentstyle=\color{codegreen},
    keywordstyle=\color{codeblue},
    numberstyle=\tiny\color{codegray},
    stringstyle=\color{codegreen},
    basicstyle=\ttfamily\footnotesize,
    breakatwhitespace=false,         
    breaklines=true,                 
    captionpos=b,                    
    keepspaces=true,                 
    numbers=left,                    
    numbersep=5pt,                  
    showspaces=false,                
    showstringspaces=false,
    showtabs=false,                  
    tabsize=2
}
 
\lstset{style=mystyle}

\setmarginsrb{3 cm}{2.5 cm}{3 cm}{2.5 cm}{1 cm}{1.5 cm}{1 cm}{1.5 cm}

\title{Lab 1}								
% Title
\author{Jayson Haddon}						
% Author
\date{08/27/2020}
% Date

\makeatletter
\let\thetitle\@title
\let\theauthor\@author
\let\thedate\@date
\makeatother

\pagestyle{fancy}
\fancyhf{}
\rhead{\theauthor}
\lhead{\thetitle}
\cfoot{\thepage}
%%%%%%%%%%%%%%%%%%%%%%%%%%%%%%%%%%%%%%%%%%%%
\begin{document}

%%%%%%%%%%%%%%%%%%%%%%%%%%%%%%%%%%%%%%%%%%%%%%%%%%%%%%%%%%%%%%%%%%%%%%%%%%%%%%%%%%%%%%%%%

\begin{titlepage}
	\centering
    \vspace*{0.5 cm}
\begin{center}    \textsc{\Large   ECE 351 }\\[2.0 cm]	\end{center}% University Name
	\textsc{\Large Introduction to Python 3.x and LATEX  }\\[0.5 cm]				% Course Code
	\rule{\linewidth}{0.2 mm} \\[0.4 cm]
	{ \huge \bfseries \thetitle}\\
	\rule{\linewidth}{0.2 mm} \\[1.5 cm]
	
	\begin{minipage}{0.4\textwidth}
		\begin{flushleft} \large
		%	\emph{Submitted To:}\\
		%	Name\\
          % Affiliation\\
           %contact info\\
			\end{flushleft}
			\end{minipage}~
			\begin{minipage}{0.4\textwidth}
            
			\begin{flushright} \large
			\emph{Submitted By :} \\
			Jayson Haddon  
		\end{flushright}
           
	\end{minipage}\\[2 cm]
	
%	\includegraphics[scale = 0.5]{PICMathLogo.png}
    
    
    
    
	
\end{titlepage}

%%%%%%%%%%%%%%%%%%%%%%%%%%%%%%%%%%%%%%%%%%%%%%%%%%%%%%%%%%%%%%%%%%%%%%%%%%%%%%%%%%%%%%%%%
\renewcommand{\thesection}{\arabic{section}}
\section{Part 1}

The cheat sheet for Spyder lists a bunch of useful key shortcuts for the program. An example of a shortcut is Cntrl + F5 runs the debug tool. 

\section{Part 2}
In Python variables do not have to be defined like they do in other coding programs. After making a variable, the print() command can be used to print the variable or anything else. In python instead of using the carrot key to square a number, you use **. Lists and arrays can be created from list or array. When creating an array or list, you can then populate it with what ever numbers you want or with all one number. Graphs are a very important tool that can be created in python. Rectangular numbers can be created in python with cRect and by using j. Python will always record the number in rectangular form. It is important to always include the imports with each new document. 

\section{Part 3}
When writing code, use spaces, tabs, and indents to make the code easier to follow. Use """ text """ to create multi line comments to help remember what the code does. Wraps lines doesn't exceed 79 characters so use brackets to extend onto different line. Use lots of comments and extra spaces around numbers. 
\section{Part 4}
Read through the latex cheat sheet and find a latex template that you want to use. Also look up addition documentation on how to insert more complex math functions. 

\section{Questions}
1.) The course that I have most enjoyed in college has been microelectronics and the courses that I am most excited for are the classes that follow microelectronics. I am also looking forward to when I don't have to go to class anymore. 

2.) The class seems fine so far, just need to get into the lab more. 
\section{Github}
https://github.com/haddjays?tab=repositories
\end{document}

%This template was created by Roza Aceska.