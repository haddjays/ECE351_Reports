%%%%%%%%%%%%%%%%%%%%%%%%%%%%%%%%%%%%%%%%%%%%%%%%%%%%%%%%%%%%%%%%
%                                                              %
% Jayson Haddon                                                %
% ECE 351 Section 51                                           %
% Lab 3                                                        %
% 02/01/2022    
%Github https://github.com/haddjays?tab=repositories
% Necessary Details:                            %
%                                                              %
%%%%%%%%%%%%%%%%%%%%%%%%%%%%%%%%%%%%%%%%%%%%%%%%%%%%%%%%%%%%%%%%
\documentclass[11pt,a4]{report}
\usepackage[english]{babel}
%\usepackage{natbib}
\usepackage{url}
\usepackage[utf8x]{inputenc}
\usepackage{amsmath}
\usepackage{graphicx}
\graphicspath{{images/}}
\usepackage{parskip}
\usepackage{fancyhdr}
\usepackage{vmargin}
\usepackage{listings}
\usepackage{hyperref}
\usepackage{xcolor}
\definecolor{codegreen}{rgb}{0,0.6,0}
\definecolor{codegray}{rgb}{0.5,0.5,0.5}
\definecolor{codeblue}{rgb}{0,0,0.95}
\definecolor{backcolour}{rgb}{0.95,0.95,0.92}

\lstdefinestyle{mystyle}{
    backgroundcolor=\color{backcolour},   
    commentstyle=\color{codegreen},
    keywordstyle=\color{codeblue},
    numberstyle=\tiny\color{codegray},
    stringstyle=\color{codegreen},
    basicstyle=\ttfamily\footnotesize,
    breakatwhitespace=false,         
    breaklines=true,                 
    captionpos=b,                    
    keepspaces=true,                 
    numbers=left,                    
    numbersep=5pt,                  
    showspaces=false,                
    showstringspaces=false,
    showtabs=false,                  
    tabsize=2
}
 
\lstset{style=mystyle}

\setmarginsrb{3 cm}{2.5 cm}{3 cm}{2.5 cm}{1 cm}{1.5 cm}{1 cm}{1.5 cm}

\title{Lab 3}								
% Title
\author{Jayson Haddon}						
% Author
\date{08/27/2020}
% Date

\makeatletter
\let\thetitle\@title
\let\theauthor\@author
\let\thedate\@date
\makeatother

\pagestyle{fancy}
\fancyhf{}
\rhead{\theauthor}
\lhead{\thetitle}
\cfoot{\thepage}
%%%%%%%%%%%%%%%%%%%%%%%%%%%%%%%%%%%%%%%%%%%%
\begin{document}

%%%%%%%%%%%%%%%%%%%%%%%%%%%%%%%%%%%%%%%%%%%%%%%%%%%%%%%%%%%%%%%%%%%%%%%%%%%%%%%%%%%%%%%%%

\begin{titlepage}
	\centering
    \vspace*{0.5 cm}
\begin{center}    \textsc{\Large   ECE 351 }\\[2.0 cm]	\end{center}% University Name
	\textsc{\Large Discrete Convolution  }\\[0.5 cm]				% Course Code
	\rule{\linewidth}{0.2 mm} \\[0.4 cm]
	{ \huge \bfseries \thetitle}\\
	\rule{\linewidth}{0.2 mm} \\[1.5 cm]
	
	\begin{minipage}{0.4\textwidth}
		\begin{flushleft} \large
		%	\emph{Submitted To:}\\
		%	Name\\
          % Affiliation\\
           %contact info\\
			\end{flushleft}
			\end{minipage}~
			\begin{minipage}{0.4\textwidth}
            
			\begin{flushright} \large
			\emph{Submitted By :} \\
			Jayson Haddon  
		\end{flushright}
           
	\end{minipage}\\[2 cm]
	
%	\includegraphics[scale = 0.5]{PICMathLogo.png}
    
    
    
    
	
\end{titlepage}

%%%%%%%%%%%%%%%%%%%%%%%%%%%%%%%%%%%%%%%%%%%%%%%%%%%%%%%%%%%%%%%%%%%%%%%%%%%%%%%%%%%%%%%%%
\tableofcontents
\pagebreak

\renewcommand{\thesection}{\arabic{section}}
\section{Introduction}
The objective of this lab was to use user defined functions to become familiar with convolutions and their properties with python. Some useful background information is to understand how a convolution functions works. A convolution works by adding each individual area of two different plots. 
\section{Equations}
The three signals used for the lab are listed below.
\begin{equation}
    f_1(t) = u(t-2) - u(t-9)
\end{equation}
\begin{equation}
    f_2(t) = e^{-t}u(t)
\end{equation}
\begin{equation}
    f_3(t) = r(t-2)[u(t-2)-u(t-3)] + r(4-t)[u(t-3)-u(t-4)]
\end{equation}
Convolution Integral
\begin{equation}
    y(t) =  \int_{0}^{p} x(p)h(t-p) \,dp 
\end{equation}
\section{Methodology}
The first task in part 1 is to use the step and ramp functions developed in the last lab to write the equations 1 through 3 as shown above. To complete this I used the step and ramp functions that I made in the previous lab. The code and plot of this can be seen below in result. 

The first part of part 2 was to create code to perform a convolution of two functions. To complete this I used the code that was given in class but I will try to explain it. The first part of the code is to take the length of the two variables that were fed into the function. The next part of the code is to create two new variables that extend the length of the two variables so that way they are the same. The next part of the code is to define a variable that is the same length of the two functions defined above and fill it with zeros. The last part of the code is to use the two new defined variables and multiply were they over lap. This is set equal to the array that was filled with zeros. This then returns the new plot. The code for this can be seen below in part 2 of the results. 

the last part of part 2 was to graph the convolution of f1 and f2, f2 and f3, and f1 and f3 and then to check each convolution with the included convolution functions that is included in numpy. The graphs and code can be seen below in the part 2 of the results. 


\section{Results}
\subsection{Part 1}
\begin{lstlisting}[language=Python]
#...........Time............#
steps = 1e-2 # Define step size
t = np.arange(0, 20 + steps , steps)



#...........Ramp and Step Function...........#
def ramp(t):
    y = np.zeros(t.shape)
    
    for i in range(len(t)):
        if (t[i] > 0):
            y[i]=t[i]
        else: 
            y[i] = 0
    return y
        
def step(t):
    x = np.zeros(t.shape)
    
    for i in range(len(t)):
        if (t[i] > 0):
            x[i]= 1
        else:
            x[i]=0
            
    return x

#...............functions to graph.......#

def f1(t):
    z = step(t-2) - step(t-9)
    
    return z
def f2(t):
    x = np.exp(-t) * step(t)
    return x
def f3(t):
    y = ramp(t-2)*(step(t-2)-step(t-3))+ramp(4-t)*(step(t-3)-step(t-4))
    return y

            
f1 = f1(t)
f2 = f2(t)
f3 = f3(t)

#.............Task 1 Plotting functions......#

plt.figure(figsize = (10, 7))
plt.subplot(3, 1, 1)
plt.plot(t,f1)
plt.grid()
plt.ylabel('Amplitude')
plt.xlabel( 't')
plt.title('f1(t)')



plt.subplot(3, 1, 2)
plt.plot(t,f2)
plt.grid()
plt.ylabel('Amplitude')
plt.xlabel( 't')
plt.title('f2(t)')



plt.subplot(3, 1, 3)
plt.plot(t,f3)
plt.grid()
plt.ylabel('Amplitude')
plt.xlabel( 't')
plt.title('f3(t)')
plt.show()
\end{lstlisting}

\begin{figure}[h!]
    \begin{center}
  \caption{Three Signals for Part 1}
  \includegraphics[scale=0.5]{3 functions.png}
\end{center}
\end{figure}
\newpage

\subsection{Part 2}
Code for Convolution 
\begin{lstlisting}[language=Python]
def convo(f1, f2):
    Nf1 = len(f1)
    Nf2 = len(f2)
    f1extend = np.append(f1, np.zeros((1,Nf2-1)))
    f2extend = np.append(f2, np.zeros((1,Nf1-1)))
    result = np.zeros(f1extend.shape)
    for i in range (Nf2 + Nf1-2):
        result[i]=0
        for j in range (Nf1):
                if(i-j+1 > 0):
                    try:
                        result[i]+= f1extend[j]*f2extend[i-j+1]
                    except:
                        print(i,j)
    return result
\end{lstlisting}

Code for all three convolution functions and checking the convolutions. 

\begin{lstlisting}[language=Python]
#................Adjusting Time............#

t = np.arange(0, 20 + steps , steps)

NT = len(t)

textend = np.arange(0, 2*t[NT-1],steps)


#..................Set up Convolution functions............#
f12 = convo(f1,f2)*steps
f23 = convo(f2,f3)*steps
f13 = convo(f1,f3)*steps


plt.figure(figsize = (10, 7))
plt.subplot(3, 1, 1)
plt.plot(textend,f12)
plt.grid()
plt.ylabel('Amplitude')
plt.xlabel( 't')
plt.title('f12')

plt.subplot(3, 1, 2)
plt.plot(textend,f23)
plt.grid()
plt.ylabel('Amplitude')
plt.xlabel( 't')
plt.title('f23')

plt.subplot(3, 1, 3)
plt.plot(textend,f13)
plt.grid()
plt.ylabel('Amplitude')
plt.xlabel( 't')
plt.title('f13')
plt.show

#................Check..............#

cf12 = scipy.signal.convolve(f1,f2)
cf23 = scipy.signal.convolve(f2,f3)
cf13 = scipy.signal.convolve(f1,f3)


plt.figure(figsize = (10, 7))
plt.subplot(3, 1, 1)
plt.plot(textend,f12)
plt.grid()
plt.ylabel('Amplitude')
plt.xlabel( 't')
plt.title('f12')

plt.subplot(3, 1, 2)
plt.plot(textend,cf12)
plt.grid()
plt.ylabel('Amplitude')
plt.xlabel( 't')
plt.title('cf12')

plt.figure(figsize = (10, 7))
plt.subplot(3, 1, 1)
plt.plot(textend,f23)
plt.grid()
plt.ylabel('Amplitude')
plt.xlabel( 't')
plt.title('f23')

plt.subplot(3, 1, 2)
plt.plot(textend,cf23)
plt.grid()
plt.ylabel('Amplitude')
plt.xlabel( 't')
plt.title('cf23')

plt.figure(figsize = (10, 7))
plt.subplot(3, 1, 1)
plt.plot(textend,f13)
plt.grid()
plt.ylabel('Amplitude')
plt.xlabel( 't')
plt.title('f13')

plt.subplot(3, 1, 2)
plt.plot(textend,cf13)
plt.grid()
plt.ylabel('Amplitude')
plt.xlabel( 't')
plt.title('cf13')

\end{lstlisting}
\newpage
\begin{figure}[h!]
    \begin{center}
  \caption{Convolution of all three functions}
  \includegraphics[scale=0.5]{convolotion of 3 functions.png}
\end{center}
\end{figure}

\begin{figure}[h!]
    \begin{center}
  \caption{f12 check}
  \includegraphics[scale=0.5]{f12 check.png}
\end{center}
\end{figure}
\newpage
\begin{figure}[h!]
    \begin{center}
  \caption{f23 check}
  \includegraphics[scale=0.5]{f23 check.png}
\end{center}
\end{figure}

\begin{figure}[h!]
    \begin{center}
  \caption{f13 check}
  \includegraphics[scale=0.5]{f13 check.png}
\end{center}
\end{figure}



\section{Error Analysis}

The difficulties I for this lab was that I didn't really know how to make the convolution function. Once I had the code shown to me, it begin to make more sense, I would have had a really hard time figuring it out on my own. 

\section{Questions}
1.) Did you work alone or with classmates on this lab? If you collaborated to get to the solution, what did the process look like?

I worked with classmates a little bit. We mostly just asked each other questions on the little things. Like making sure our three functions were the same and question about plotting the functions. 

2.) What was the most difficult part of this lab for you, and what did you problem-solving process look like? 

The most difficult part of this lab for me was to understand how to go about creating the convolution function. I still have a hard time understanding how arrays work in coding. This lack of knowledge on arrays created some difficulty understanding how you can set the length of two arrays equal to each other by filing in the empty spaces with zeros and then how to individually add each part of an array. 


3.) Did you approach writing the code with analytical or graphical convolution in mind? Why did you chose this approach?

 I initially approached this lab with the graphical convolution in mind, but after getting a little help on the lab, I then approached it with analytical convolution. Coding wise, it was easier to think about the sum of each individual piece being added together then it was to think about two plots being put together. 
 
 4.) Leave any feedback on the clarity of the lab tasks, expectations, and deliverables.
 
 The instructions of this lab was helpful. It told you want you needed to do and acomplish. 


\section{Conclusion}
From this lab, I got a way better understanding of how a convolution works. It was very useful to have the explanation about the smoke in the air from matches and then to be able to see how the new plot is the result of both the two old plots being multiplied together. It was also helpful to see create three new functions with the ramp and step and then take the convolution of them. I even used this code to help me check my last homework in class. 



\section{Github}
https://github.com/haddjays?tab=repositories
\end{document}

\section{Conclusion}

%This template was created by Roza Aceska.
