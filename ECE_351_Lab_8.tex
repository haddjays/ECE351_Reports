%%%%%%%%%%%%%%%%%%%%%%%%%%%%%%%%%%%%%%%%%%%%%%%%%%%%%%%%%%%%%%%%
%                                                              %
% Jayson Haddon                                                %
% ECE 351 Section 51                                           %
% Lab 8                                                        %
% 03/22/2022    
%Github https://github.com/haddjays?tab=repositories
% Necessary Details:                            %
%                                                              %
%%%%%%%%%%%%%%%%%%%%%%%%%%%%%%%%%%%%%%%%%%%%%%%%%%%%%%%%%%%%%%%%
\documentclass[11pt,a4]{report}
\usepackage[english]{babel}
%\usepackage{natbib}
\usepackage{url}
\usepackage[utf8x]{inputenc}
\usepackage{amsmath}
\usepackage{graphicx}
\graphicspath{{images/}}
\usepackage{parskip}
\usepackage{fancyhdr}
\usepackage{vmargin}
\usepackage{listings}
\usepackage{hyperref}
\usepackage{xcolor}
\definecolor{codegreen}{rgb}{0,0.6,0}
\definecolor{codegray}{rgb}{0.5,0.5,0.5}
\definecolor{codeblue}{rgb}{0,0,0.95}
\definecolor{backcolour}{rgb}{0.95,0.95,0.92}

\lstdefinestyle{mystyle}{
    backgroundcolor=\color{backcolour},   
    commentstyle=\color{codegreen},
    keywordstyle=\color{codeblue},
    numberstyle=\tiny\color{codegray},
    stringstyle=\color{codegreen},
    basicstyle=\ttfamily\footnotesize,
    breakatwhitespace=false,         
    breaklines=true,                 
    captionpos=b,                    
    keepspaces=true,                 
    numbers=left,                    
    numbersep=5pt,                  
    showspaces=false,                
    showstringspaces=false,
    showtabs=false,                  
    tabsize=2
}
 
\lstset{style=mystyle}

\setmarginsrb{3 cm}{2.5 cm}{3 cm}{2.5 cm}{1 cm}{1.5 cm}{1 cm}{1.5 cm}

\title{Lab 8}								
% Title
\author{Jayson Haddon}						
% Author
\date{08/27/2020}
% Date

\makeatletter
\let\thetitle\@title
\let\theauthor\@author
\let\thedate\@date
\makeatother

\pagestyle{fancy}
\fancyhf{}
\rhead{\theauthor}
\lhead{\thetitle}
\cfoot{\thepage}
%%%%%%%%%%%%%%%%%%%%%%%%%%%%%%%%%%%%%%%%%%%%
\begin{document}

%%%%%%%%%%%%%%%%%%%%%%%%%%%%%%%%%%%%%%%%%%%%%%%%%%%%%%%%%%%%%%%%%%%%%%%%%%%%%%%%%%%%%%%%%

\begin{titlepage}
	\centering
    \vspace*{0.5 cm}
\begin{center}    \textsc{\Large   ECE 351 }\\[2.0 cm]	\end{center}% University Name
	\textsc{\Large Fourier Series Approximation of a Square Wave  }\\[0.5 cm]				% Course Code
	\rule{\linewidth}{0.2 mm} \\[0.4 cm]
	{ \huge \bfseries \thetitle}\\
	\rule{\linewidth}{0.2 mm} \\[1.5 cm]
	
	\begin{minipage}{0.4\textwidth}
		\begin{flushleft} \large
		%	\emph{Submitted To:}\\
		%	Name\\
          % Affiliation\\
           %contact info\\
			\end{flushleft}
			\end{minipage}~
			\begin{minipage}{0.4\textwidth}
            
			\begin{flushright} \large
			\emph{Submitted By :} \\
			Jayson Haddon  
		\end{flushright}
           
	\end{minipage}\\[2 cm]
	
%	\includegraphics[scale = 0.5]{PICMathLogo.png}
    
    
    
    
	
\end{titlepage}

%%%%%%%%%%%%%%%%%%%%%%%%%%%%%%%%%%%%%%%%%%%%%%%%%%%%%%%%%%%%%%%%%%%%%%%%%%%%%%%%%%%%%%%%%
\tableofcontents
\pagebreak

\renewcommand{\thesection}{\arabic{section}}
\section{Introduction}
The purpose of this lab was to use Fourier series to approximate a square wave. 
\section{Equations}
The three equations for a fourier series is shown below.
\begin{equation}
    x(t) = \frac{1}{2}*a_0+  \sum_{k=1}^{\infty} a_kcos(kw_0t) + b_ksin(kw_0t)
\end{equation}
\begin{equation}
    a_k =  \frac{2}{T} \int_{0}^{T} x(t)cos(kw_0t) \,dt 
\end{equation}
\begin{equation}
    b_k =  \frac{2}{T} \int_{0}^{T} x(t)sin(kw_0t) \,dt 
\end{equation}
The Fourier series for the square wave is 
\begin{equation}
    b_k= \frac{2}{k\pi}*(1-cos(k\pi))
\end{equation}
\begin{equation}
    x(t) = \sum_{k=1}^{\infty} \frac{2}{k\pi}*(1-cos(k\pi))*sin(k*\frac{2\pi}{T}*t)
\end{equation}

\section{Methodology}
The first task in part one was to create the expressions for ak and bk in python and solve for a0, a1, b1, b2, and b3. To do this I create two user defined functions. The first one was for ak and took the value of k. This just returned zero because there was no ak in the odd square wave. The second was for bk and took the value of k. It then set x equal to equation 4 and returned x. I then called each function for each value that was asked for. The code and output of this can be seen below in the results part 1 task 1.

The second task of part one was to plot the Fourier series approximation for N = 1,3,15,50,150, and 1500 for T =8s. To complete this I created a user defined function to take the value N. I then created two dummy variable that I call aksum and bksum. I then used the expression for num in np.arange(1, N+1) to set the variable num equal to the ak or bk function I created in the first task. I then set the aksum or bksum equal to num plus the old aksum or bksum. This would do this for however many times the N was set for which would then be used to plot the Fourier series. The code and the 6 figures can be seen below in the results. 

\section{Results}


\subsection{Part 1 Task 1}

\begin{lstlisting}[language=Python]
def ak(k):
    x = 0
    return x


def bk(k):
    x = (2/(k*math.pi))*(1-np.cos(k*math.pi))
    return x

print('Ak of 0 =', ak(0) )
print('Ak of 1 =', ak(1) )
print('Ab of 1 =', bk(1) )
print('Ab of 2 =', bk(2) )
print('Ab of 3 =', bk(3) )

\end{lstlisting}

\begin{lstlisting}
Ak of 0 = 0
Ak of 1 = 0
Ab of 1 = 1.2732395447351628
Ab of 2 = 0.0
Ab of 3 = 0.4244131815783876
\end{lstlisting}

\subsection{Part 1 Task 2}

\begin{lstlisting}
def x(N):
    T = 8
    w0 = (2*math.pi)/T
    aksum = 0
    bksum = 0
    for num in np.arange(1,N+1):
        num = ak(num)*np.cos(num*w0*t)
        aksum = aksum + num
        #print('AKSum =', aksum)
    for num in np.arange(1,N+1):
        num = bk(num)*np.sin(num*w0*t)
        bksum = bksum + num
        #print('BKSum =', bksum)
    x = 1/2*0 + aksum+bksum
    return x

#print('bk =',x(t,5) )

plt.figure(figsize = (10, 12))
plt.subplot(3, 1, 1)
plt.plot(t,x(1))
plt.grid()
plt.ylabel('Amplitude ')
plt.xlabel( 't')
plt.title('N = 1')

plt.subplot(3, 1, 2)
plt.plot(t,x(3))
plt.grid()
plt.ylabel('Amplitude ')
plt.xlabel( 't')
plt.title('N = 3')

plt.subplot(3, 1, 3)
plt.plot(t,x(15))
plt.grid()
plt.ylabel('Amplitude ')
plt.xlabel( 't')
plt.title('N = 15')

plt.figure(figsize = (10, 12))
plt.subplot(3, 1, 1)
plt.plot(t,x(50))
plt.grid()
plt.ylabel('Amplitude ')
plt.xlabel( 't')
plt.title('N = 50')

plt.subplot(3, 1, 2)
plt.plot(t,x(150))
plt.grid()
plt.ylabel('Amplitude ')
plt.xlabel( 't')
plt.title('N = 150')

plt.subplot(3, 1, 3)
plt.plot(t,x(1500))
plt.grid()
plt.ylabel('Amplitude ')
plt.xlabel( 't')
plt.title('N = 1500')
plt.show
\end{lstlisting}

\begin{figure}[h!]
    \begin{center}
  \caption{Fourier series for N = 1,3,15}
  \includegraphics[scale=0.7]{N 1,3,15.png}
\end{center}
\end{figure}
\newpage

\begin{figure}[h!]
    \begin{center}
  \caption{Fourier series for N = 50,150, 1500}
  \includegraphics[scale=0.7]{N 50,150, 1500.png}
\end{center}
\end{figure}

\section{Error Analysis}
 The hardest part of this lab for me was getting the summation of the Nth term to work. To get the summation part to work I looked online and had to go through a few testing stages where I used some code that someone made, and adjusted it until I thought it was working how I wanted it. Then I would plug in my ak and bk functions to see if it would work correctly. 
\section{Questions}
1. Is x(t) an even or an odd function? Explain why.

x(t) is an odd function. This is because x(-t) = -x(t) which means that the function acts like a sine wave.

2. Based on your results from Task 1, what do you expect the values of a2, a3, . . . , an to be? Why?

Since this function is an odd function there is no cosine part in the function so an will always be zero. 

3. How does the approximation of the square wave change as the value of N increases? In what
way does the Fourier series struggle to approximate the square wave?

The approximation of the square wave gets closer and closer to a square wave as N gets bigger. This is because our Fourier series is based just on a sine wave, so at lower N values, the graph looks like a sine wave, but as you add more and more iterations of the Fourier series it begins to look like the function it was modelled after, which is the square wave. The Fourier series struggles to approximate the part where the square goes from straight vertical to straight horizontal. 

4. What is occurring mathematically in the Fourier series summation as the value of N increases?

For each N value there is the ak and bk value that is being evaluated at that N value. Each of these N values are then added together to make the function look more and more like it originally did. For this lab as each value of N is being added together, the function over time looks more and more like a square wave. 

5. Leave any feedback on the clarity/usefulness of the purpose, deliverables, and expectations
for this lab.

I have no feedback for this lab. 

\section{Conclusion}
In conclusion, this lab was really useful. It was cool to see how you could get an expression for a function. Take the Fourier series of that function and take the summation of that Fourier series at different N values. It was very useful to see how as you increased the N value you got closer to the actual functions that you took the Fourier series of. 


\section{Github}
https://github.com/haddjays?tab=repositories
\end{document}

\section{Conclusion}

%This template was created by Roza Aceska.
